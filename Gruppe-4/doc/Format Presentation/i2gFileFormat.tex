\documentclass[11pt]{beamer}
\usepackage[utf8]{inputenc}
\usepackage[T1]{fontenc}


\usetheme{default}
\usepackage{listings}
\lstset{language=xml, frame=single}

\begin{document}
	%\author{Altenbernd, Braungardt, Thalheim}
	\title{Intergeo File Format .i2g}
	%\subtitle{}
	%\logo{}
	%\institute{}
	\date{03.05.2018}
	%\subject{}
	%\setbeamercovered{transparent}
	%\setbeamertemplate{navigation symbols}{}
	\frame[plain]{\maketitle}
	
	\begin{frame}%[fragile]
		\frametitle{Generelle Struktur}
		\noindent
		Zwei Parts:
		\begin{description}
			\item[statisch:] xml-Struktur\\
			atomare Variablen
			\item[dynamisch:] Elemente (Punkte, Linien, aber auch Buttons etc.)\\
			Konstruktionen\\
			Anzeigeoptionen (Farben, etc.)
		\end{description}
	\end{frame}

	\begin{frame}[fragile]
		\frametitle{Generelle Struktur}
		\begin{lstlisting}
<xs:element name="construction">
  <xs:complexType>
    <xs:sequence>
      <xs:element ref="elements"/>
      <xs:element ref="constraints"/>
      <xs:element ref="display" minOccurs="0" 
                  maxOccurs="unbounded"/>
    </xs:sequence>
  </xs:complexType>
</xs:element>
		\end{lstlisting}
	\end{frame}

	\begin{frame}[fragile]
		\frametitle{Elemente}
		Zum Beispiel ein euklidischer Punkt:
		\begin{lstlisting}
<elements>
  <point id="A">
    <euclidian_coordinates>
      <double>3.55</double>
      <double>-4</double>
      <double>0</double>
    </euclidian_coordinates>
  </point>
</elements>
		\end{lstlisting}
	\end{frame}

\begin{frame}
	\frametitle{Elemente}
	\begin{itemize}
		\item \texttt{point}
		\item \texttt{line} $(2,3,5)\to 2x+3y+5=0$
		\item \texttt{line\_segment}
		\item \texttt{ray} (unendliche Linie)
		\item \texttt{polygon}
		\item \texttt{vector}
		\item \texttt{conic}
		\item \texttt{circle}
		\item \texttt{ellipse}
		\item \texttt{parabola}
		\item \texttt{hyperbola}
		\item \texttt{locus}
	\end{itemize}
\end{frame}


	\begin{frame}[fragile]
		\frametitle{Konstruktion}
		Zum Beispiel eine Gerade durch zwei Punkte als Vorschrift:
		\begin{lstlisting}
<constraints>
  <line_through_two_points>
    <line out="true">|</line>
    <point>A</point>
    <point>B</point>
  </line_through_two_points>
<constraints>
		\end{lstlisting}
	\end{frame}

\begin{frame}
	\frametitle{Konstruktionen}
	Zum Beispiel möglich:
	\begin{itemize}
		\item \texttt{midpoint\_of\_two\_points}
		\item \texttt{point\_intersection\_of\_two\_lines}
		\item \texttt{midpoint\_of\_line\_segment}
		\item \texttt{vector\_of\_ray}
		\item \texttt{circle\_by\_center\_and\_point}
		\item \texttt{circle\_by\_three\_points} (mit Degenerierung: gleiche Punkt, Punkte auf gemeinsamer Geraden)
		\item \texttt{symmetry\_by\_line}
	\end{itemize}
\end{frame}

	\begin{frame}[fragile]
		\frametitle{Darstellung}
		Zum Beispiel Darstellung des Punktes A:
		\begin{lstlisting}
<display>
  <style ref="A">
    <label>The point A</label>
    <fill>#ffffff</fill>
    <stroke>#222222</stroke>
    <stroke_width>2</stroke_width>
    <point_size>5</point_size>
  </style>
</display>
		\end{lstlisting}
\end{frame}

\begin{frame}
	\frametitle{Darstellung}
	Typische Eigenschaften:
	\begin{itemize}
		\item \texttt{stroke}
		\item \texttt{stroke-width}
		\item \texttt{borderwidth}
		\item \texttt{fill}
		\item \texttt{fill-opacity}
		\item \texttt{point-size}
		\item \texttt{point-style}
		\item \texttt{label}
		\item \texttt{visible}
		\item \texttt{background-color}
	\end{itemize}
\end{frame}

\end{document}